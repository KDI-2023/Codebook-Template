\subsection{Scope Definition}

As the main place of living and the main carrier of "living", which is one of the four major human behaviors of clothing, food, housing and transportation, houses provide people with the material basis of basic living while carrying rich cultural connotations such as architectural culture, living etiquette culture and family culture. As one of the basic needs of human beings, the purchase of a house has become a common issue in our society today. The demand for education has led to the hot demand for school district houses; the superstition of feng shui makes people think twice about the house type before buying a house; besides, the demand for traffic around the house also varies from person to person; the different economic conditions also make the expected area of the house vary greatly; the expectation of future life and other personalized needs for buying a house have also become important factors to be considered.


Based on this, the project aims to provide a unified, comprehensive and structured query solution for the query by cleaning, organizing and fusing the official and authoritative data from the Internet based on the knowledge graph method through the inputted needs related to home purchase (e.g. price range, personality needs, house type consideration). The solution can effectively provide an effective and reasonable query structure for home purchase needs that meet the requirements of the paradigm, which helps reduce the customer's decision space and provides a key theoretical foundation and technical support for improving people's living standards and happiness.


Our data is sampled from the 58 Tongcheng classified information service integration platform, which can provide home buyers with authoritative, reliable, complete and comprehensive information on the basis of home buying. We narrow down the data set by calling the search interface of the website to query specific data, use python scripts to sample the website data and clean it to make it structured; we construct our heterogeneous database group by sampling and cleaning under different search restrictions and reasonably discarding the features of the result data, which is the data source of the project.




Our project is to be implemented according to the following steps:
~\\
\item
 \textbf{Informal Modeling}. This section is dedicated to the Informal Modeling phase description.  The Section is divided in Schema and Data level in order to report the details of the elements involved in the generation of the schema, as well as the description of the datasets evolution in this phase.  Moreover a specif section, one for each level, reports the difference between the elements defined in this phase and the definitions in the previous phase, analyzing in this way the variance in the different phase.
 ~\\
 \item
 \textbf{Formal Modeling}. This section is dedicated to the Formal Modeling phase description.  The Section is divided in Schema and Data level in order to report the details regarding both the ontology generated and the datasets version in the current phase.
 ~\\
 \item  
 \textbf{Data integration}. This section is dedicated to the Data Integration phase.